\chapter{Topics to be aware of}
\label{chp:howto}

\section{Permissions}
\label{chp:howto:sec:permissions}

As the Android OS is built on top of the Linux Kernel it also comes with the permission approach of Linux. This way an application can only access a limited range of system resources by default and every access to a resource is managed by the OS.
To get more access than the standard provided by the basic sandbox, an application must define the resources it wants to have access to in its manifest. The user gets asked to grant these permissions the first time an app wants the access. This gets saved to the device for later usage and the user doesn't have to grant it again.
Like in Linux the permission model is a user based model and as every application is its own user every application has its own permissions. This also isolates the user resources from one another. In addition every application has to explicitly define which resources it shares with other applications.
A good way of requesting permissions is to minimize the number of requested permissions. Simply because if an app can't do more than it should, unexpected situations won't arise. So if a permission is not required it should not be requested.
Also self created permissions should be as few as possible and rather system defined ones should be used as while granting the permissions a user could get confused by a big list of unknown permissions.

\section{Networking}
\label{chp:howto:sec:networking}

Networking with any device is always risky just because of the fact, that the data that has to be transmitted is potentially private data of the user. Any loss or "publication" of this data can harm the user and/or the trust a user has in the application. The highest endeavour should always be to keep user data secure at all times. For this reason it is important to use secure connections to any network an app wants to send data to or receive from. 
The key to secure network traffic is to use appropriate protocols for the connections. For trivial example would be to use HTTPS over HTTP whenever the server provides it.
On a mobile device any network connection covers an additional possible security threat, as it is frequently connected to unsecure networks via Wi-Fi. Here the threat is first that the network itself could be unsecure and second it is not known which other users are in the network and possibly have bad intentions.
Another point is that some developer tend to use localhost ports for sending data over Inter Process Communication (IPC). This is not a good approach as these interfaces are accessible by other applications on the device and so the data could be read by the wrong process. The way to solve this is to use IPC mechanisms provided by android.

\section{Input Validation}
\label{chp:howto:sec:inputValidation}

This topic is the most common security problem [https://developer.android.com/training/articles/security-tips] if not done correctly or not done at all. Every data an app is receiving over the network, as input from the user or from an IPC is potentially threatening even if it is not meant to be harmful in first place.
The most common problems are buffer overflows, use after free and off-by-one errors.[] This threat can be reduced if the data gets validated.
Type-safe languages already tend to reduce the likelihood of input validation issues. Also pointers should always be handled very carefully so that they don't point on the wrong address and when using buffers they should always be managed.
When using queries for an SQL database there could be the issue of SQL injection that should be taken care of by using parameterized queries or limiting permissions to read-only or write-only. Another way is to use well formated data formats and verify each expected format.

\section{Handeling User Data}
\label{chp:howto:sec:userData}

Nowadays users get more and more aware that their personal data is important and that it should be handled with care. For this it is necessary to not lose the trust of the user while using an app because he wouldn't come back to it then anymore.
One good approach to accomplish this to minimize the use of APIs where the data is sent to. If this is necessary for the functionality of the app the way to go would be to not send the plain information but a hash or a non-reversible form of the data. This way the data doesn't get exposed to third parties for services or advertisement. This approach is also valid when storing the data on the phone.
The data could also get leaked to other apps through permissions to IPCs meaning that one app has the data and another app normaly has no permission for accessing this data but it gets it through a connection to the first app.
Third party services or advertisement often require personal information of users to work for not obvious reasons. Here a good thing to do is not to provide that data if the developer is not sure why the service would need that information.

\section{Cryptography}
\label{chp:howto:sec:cryptography}

Next to other security features that were mentioned, Android also provides a wide array of algorithms for protecting data using cryptography. So there is no need to implement your own algorithms.
A good approach for this is to always use the highest level of the existing implementations that still support the usecase in the app. This way the developer can make the data as save as possible because the levels of security differentiate in how strong they are.
While using these algorithms it is also important to use secure random number generators to initialize these algorithms. If in this case the developer only uses a non secure random number generator this weakens the cryptographic algorithms significantly.

\section{Credentials}
\label{chp:howto:sec:credentials}

A good approach for handeling user credentials is to minimize the frequency of asking for the user credentials because asking for it a lot makes phishing attacks more likely to be successful each time.
What is quite obvious is that the developer should never store user names or passwords on the device itself. Instead the way to go should be to initially authenticate a user and then use an authorization token.

\section{Interprocess Communication}
\label{chp:howto:sec:interprocessCommunication}

This communication appears when sending data from one app to another or also from process to process within an app.
For this Android provides a functionality using intents. These intents get sent indirectly to another app via Reference Monitor which takes care of the security in this case. When the sending process or application specifies a permission that is needed to receive the intent the Reference Monitor checks if the receiver actually has this permission and if not it cancels the communication. So the permission gets mandatory to receive this specific intent.
These intents can be broadcasted so that every process has the chance to get it or sent directly to a receiver. If the data that has to be sent is sensitive the better approach would be to send it directly for obvious reasons.

\section{Dynamically Loaded Code}
\label{chp:howto:sec:dynamicallyLoadedCode}

This is loading executable code from somewhere else which is obviously a security risk as the source is not always fully transparent and so it may be harmful.
It is a big threat for code injection or code tampering which then can do harmful activities as it eather adds or manipulates the code of the apk.
Dynamically loaded code can make it impossible to verify the behaviour of an app which then in the next step can make it prohibited in some environments as the code is more or less unpredictable.
One important thing to keep in mind is that dynamically loaded code runs with the same security permissions as the app itself.
