\chapter{Topics to be aware of}
\label{chp:howto}

\section{Permissions}
\label{chp:howto:sec:permissions}

As the Android OS is built on top of the Linux Kernel it also comes with the permission approach of Linux. This way an application can only access a limited range of system resources by default and every access to a resource is managed by the OS.
To get more access than the standard provided by the basic sandbox, an application must define the resources it wants to have access to in its manifest. The user gets asked to grant these permissions the first time an app wants the access. This gets saved to the device for later usage and the user doesn't have to grant it again.
Like in Linux the permission model is a user based model and as every application is its own user every application has its own permissions. This also isolates the user resources from one another. In addition every application has to explicitly define which resources it shares with other applications.
A good way of requesting permissions is to minimize the number of requested permissions. Simply because if an app can't do more than it should, unexpected situations won't arise. So if a permission is not required it should not be requested.
Also self created permissions should be as few as possible and rather system defined ones should be used as while granting the permissions a user could get confused by a big list of unknown permissions.

\section{Networking}
\label{chp:howto:sec:networking}

Networking with any device is always risky just because of the fact, that the data that has to be transmitted is potentially private data of the user. Any loss or "publication" of this data can harm the user and/or the trust a user has in the application. The highest endeavour should always be to keep user data secure at all times. For this reason it is important to use secure connections to any network an app wants to send data to or receive from. 
The key to secure network traffic is to use appropriate protocols for the connections. For trivial example would be to use HTTPS over HTTP whenever the server provides it.
On a mobile device any network connection covers an additional possible security threat, as it is frequently connected to unsecure networks via Wi-Fi. Here the threat is first that the network itself could be unsecure and second it is not known which other users are in the network and possibly have bad intentions.
Another point is that some developer tend to use localhost ports for sending data over Inter Process Communication (IPC). This is not a good approach as these interfaces are accessible by other applications on the device and so the data could be read by the wrong process. The way to solve this is to use IPC mechanisms provided by android.

\section{Input Validation}
\label{chp:howto:sec:inputValidation}

This topic is the most common security problem [https://developer.android.com/training/articles/security-tips] if not done correctly or not done at all. Every data an app is receiving over the network, as input from the user or from an IPC is potentially threatening even if it is not meant to be harmful in first place.
The most common problems are buffer overflows, use after free and off-by-one errors.[] This threat can be reduced if the data gets validated.
Type-safe languages already tend to reduce the likelihood of input validation issues. Also pointers should always be handled very carefully so that they don't point on the wrong address and when using buffers they should always be managed.
When using queries for an SQL database there could be the issue of SQL injection that should be taken care of by using parameterized queries or limiting permissions to read-only or write-only. Another way is to use well formated data formats and verify each expected format.
